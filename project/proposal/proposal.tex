\documentclass[11pt]{article}

% ==== PACKAGES ==== %
% \usepackage{fullpage}
\usepackage{amsmath,amssymb,amsthm}
\usepackage{epic}
\usepackage{eepic}
\usepackage{hyperref}
\usepackage{listings}
\usepackage{float}
\usepackage{graphicx}
\usepackage{subcaption}
\usepackage{fancyhdr}
\usepackage{color}
\usepackage{bbm}
\usepackage[letterpaper, margin=1in]{geometry}

% ==== MARGINS ==== %
% \pagestyle{empty}
% \setlength{\oddsidemargin}{0in}
% \setlength{\textwidth}{6.8in}
% \setlength{\textheight}{9.5in}

\pagestyle{fancy}
\fancyhf{}
\rhead{CSCI 5229}
\lhead{Project Proposal}
\rfoot{Page \thepage}


\newtheorem*{solution*}{Solution}
\newtheorem{lemma}{Lemma}[section]
\newtheorem{theorem}[lemma]{Theorem}
\newtheorem{claim}[lemma]{Claim}
\newtheorem{definition}[lemma]{Definition}
\newtheorem{corollary}[lemma]{Corollary}
\lstset{moredelim=[is][\bfseries]{[*}{*]}}

% ==== DOCUMENT PROPER ==== %
\begin{document}

\thispagestyle{empty}

% --- Header Box --- %
\newlength{\boxlength}\setlength{\boxlength}{\textwidth}
\addtolength{\boxlength}{-4mm}

\begin{center}\framebox{\parbox{\boxlength}{\bf
      Computer Graphics \hfill Project Proposal\\
      CSCI 5229 Fall 2018 \hfill Due Date: Oct 24, 2018\\
      Name: Andrew Kramer \hfill 
}}
\end{center}

\section*{Project Theme}
This project will focus on creating a graphical frontend for a simultaneous localization and mapping program. It will render a vehicle that will move through a simulated environment, tracking the movement of a real-world robot estimated by the SLAM system. The simulated vehicle will accurately track both the location and orientation of the vehicle being tracked by the SLAM system. The environment will be consist of a sky box and the landmarks being used to estimate the vehicle's position. Landmarks will appear when they are added to the SLAM problem and they will fade (but not disappear) when they are no longer active in the problem.  Also, because running a SLAM system on live sensors will not be practical for demonstration purposes, it will include the ability to run using logged sensor data.

\section*{Deliverables}

\subsection*{Required}
\begin{itemize}
\item Simulated vehicle accurately tracks estimated pose from SLAM or log file.
\item Simulated vehicle stays within the environment's skybox.
\item Simulated vehicle has accurate lighting and shadows.
\item Scene can be reoriented and resized by the user using the mouse as in most CAD programs.
\item Landmarks accurately track the estimated positions of landmarks in the physical world.
\item User can toggle the display of previous poses.
\item User can toggle the display of active landmarks in the SLAM problem.
\item User can toggle between vehicle-centered and environment-centered views.
\item User can toggle between perspective and orthogonal projections. 
\item User options use onscreen buttons rather than keyboard commands.
\end{itemize}

\subsection*{Extra}
\begin{itemize}
\item User can toggle landmark labels
\item User can adjust the speed of the visualization when running from logged data
\item Landmarks observed in the current frame are connected to the simulated vehicle with a transparent ray
\item User can switch mode to first-person cockpit view, allowing user to fly around with the virtual SLAM vehicle.
\end{itemize}

\end{document}
